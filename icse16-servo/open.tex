%!TEX root = paper.tex

\section{Open problems}
\label{sec:open}
While this work has discussed many challenges in browser design and our current progress,
there are many other interesting open problems.

\subsection{Just-in-time code} JavaScript engines dynamically produce native code that is
intended to execute more efficiently than an interpreted strategy.
Unfortunately, this area is a large source of security bugs.
These bugs come from two sources.
First, there are potential correctness issues.
Many of these optimizations are only valid when certain conditions of the calling
code and environment hold, and ensuring the specialized code is called only when those
conditions hold is non-trivial.
Second, dynamically producing and compiling native code and patching it into memory
while respecting all of the invariants required by the JavaScript runtime (e.g., the
garbage collector's read/write barriers or free vs. in-use registers) is also a challenge.

\subsection{Unsafe code correctness} Today, when we write unsafe code in Rust
there is limited validation of memory lifetimes or type safety within that
code block.
However, many of our uses of unsafe code are well-behaved translations of
either pointer lifetimes or data representations that cannot be annotated
or inferred in Rust.
We are very interested in additional annotations that would help us prove
basic properties about our unsafe code, even if these annotations require a
theorem prover or ILP solver to check.

\subsection{Incremental computation} As mentioned in \secref{sec:browsers},
all modern browsers use some combination of dirty bit marking and incremental
recomputation heuristics to avoid reprocessing the full page when
a mutation is performed.
Unfortunately, these heuristics are not only frequently the source of
performance differences between browsers, but they are also a source of
correctness bugs.
A library that provided a form of self adjusting computation suited to
incremental recomputation of only the visible part of the page, perhaps
based on the Adapton~\cite{adapton} approach, seems promising.

%%% Local Variables: 
%%% mode: latex
%%% TeX-master: "paper"
%%% End: 
