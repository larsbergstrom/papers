%!TEX root = paper.tex
\section{Introduction}
\label{sec:intro}
When most web browsers were originally designed, web pages were mostly static.
Modern web browsers do not just display static pages, but run web applications
with complexity similar to native software.
From application suites such as Google Apps\footnote{\url{https://apps.google.com}} to games
based on the Unreal Engine,\footnote{\url{https://www.unrealengine.com/}}
modern browsers are a delivery platform for the types of rich media experiences historically
tied to single hardware and operating system platforms.
This shift has greatly increased the amount and complexity of code in a modern web engine
as well as users' expectations around performance and security.

The heart of a modern web browser is its engine, the code responsible
for loading, processing, evaluating, and rendering web content.
There are three major browser engine families:
\begin{enumerate}
\item Trident/Spartan, the engine in Internet Explorer~\cite{IE}
\item Webkit\cite{WEBKIT}/Blink, the engine in Safari~\cite{SAFARI}, Chrome~\cite{CHROME}, and Opera~\cite{OPERA}
\item Gecko, the engine in Firefox~\cite{FIREFOX}
\end{enumerate}
All of these engines have at their core many millions of lines of \Cplusplus{} code.
The use of \Cplusplus{} has enabled all of these browsers to achieve excellent sequential
performance on a single web page, particularly on desktop computers.
But, they all face several challenges:
\begin{itemize}
  \item On mobile devices with lower processor speed but many more processors, these browsers
    do not provide the same level of interactivity~\cite{parallelizing-web-pages,ZOOMM}.
  \item In Gecko, we determined that roughly 50\% of the security critical bugs are memory use after
    free, array out of range access, or related to integer overflow.
  \item As the web has become more interactive, the mostly-sequential architecture of these engines
    has made it challenging to incorporate new features without sacrificing interactivity.
  \item With the growth in the popularity of other languages at the expense of \Cplusplus{}, both
    the number of volunteer contributors to the core \Cplusplus{} parts of these browser engine
    open source codebases has not grown apace with the increase in the size
    of the codebase.
\end{itemize}

Servo~\cite{SERVO} is a new web browser engine designed to address the major environment and 
architectural changes over the last decade.
The goal of the Servo project is to produce a browser that enables new applications to be authored
against the web platform that run with more safety, better performance, and better power usage
than in current browsers.

To address memory-related safety issues, we are using a new systems programming language,
Rust~\cite{RUST}.
In Rust, errors such as memory buffer use after free and data races are prevented statically at
compile time.

For parallelism and power, we scale across a wide variety of hardware by building either data-
or task-parallelism, as appropriate, into each part of the web platform.
Additionally, we are improving concurrency by reducing the simultaneous access to data
structures and using a message-passing architecture between components such as the
JavaScript engine and the rendering engine that paints graphics to the screen.

With an average of 5 new contributors per week and several volunteers who have turned
into key members of the project, we believe that Rust has helped Servo to lower the
barrier to entry in systems programming.

Servo is currently over 800k lines of Rust code and implements enough of the web platform to render and
process many pages, though it is still a far cry from the over 7 million lines of code in
the Mozilla Firefox browser.
However, we believe that we have implemented enough of the web platform to provide an
early report on the successes, failures, and open problems remaining in Servo, from the
point of view of programming languages and runtime research.
In this experience report, we discuss the design and architecture of a modern web 
browser engine, show how using the Rust programming language has helped us to  address the
engineering challenges we have encountered when building the browser engine, and also touch on
open problems and areas of future investigation.

%%% Local Variables: 
%%% mode: latex
%%% TeX-master: "paper"
%%% End: 
