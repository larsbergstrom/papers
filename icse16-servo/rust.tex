%!TEX root = paper.tex

\section{Rust}
\label{sec:rust}

Rust is a statically typed systems programming language most heavily inspired by the C
and ML families of languages~\cite{RUST}.
Like the C family of languages, it gives the developer fine control over memory layout
and predictable performance.
Unlike C programs, Rust programs are \emph{memory safe} by default,
only allowing unsafe operations in specially-delineated \lstinline[language=Rust]{unsafe} blocks.

Beyond performance and safety Rust features the expressive facilities
of modern high level languages such as generics, algebraic data types,
pattern matching, and closures.
Abstractions in Rust are designed to have predictable and minimal
performance impact, approaching the ideal of 'zero-cost' abstractions.

Complete documentation and a language reference for Rust are available at: \url{http://doc.rust-lang.org/}.

\subsection{Feature overview}

Rust's syntax draws heavily from C++, and many of its features will be
familiar to modern programmers, though the details of both
often differ in significant and interesting ways.
What follows is a brief primer of the basics needed to understand the examples in this paper.

Local variables, declared with \lstinline[language=Rust]{let}, are immutable by default.
To mutate a variable in Rust it must be declared \lstinline[language=Rust]{mut}.
Both scalars and aggregates like structs are value types, allocated on the stack.
The heap is accessed through library container types, the simplest of which are \lstinline[language=Rust]{Box},
a pointer to a value on the heap, and \lstinline[language=Rust]{Vec}, a dynamically-sized
array of values. Both mutability and simple heap types are
illustrated in \figref{fig:mut}.

\begin{figure}
\begin{lstlisting}
let max = 100;
let mut counter = 0;
let mut boxes: Vec<Box<i32>> = Vec::new();
while counter < max {
  // Append a boxed integer to a vector
  boxes.push(Box::new(counter));
  counter += 1;
}
\end{lstlisting}
  \caption{Mutability and heap allocation.}
  \label{fig:mut}
\end{figure}

Rust's most prominent influence from the ML languages are algebraic data types,
along with \emph{pattern matching} to destructure
them into their components. Rust calls algebraic data types \emph{enums},
and they are the union of multiple types,
each of which contains their own fields. Although superficially
similar to C unions, they are significantly different in that
instances of enums cannot be indiscriminately cast between
variants; to access the values in an enum one must employ the \lstinline[language=Rust]{match}
expression to first check the variant, then bind references
to its interior fields. In \figref{fig:enums} the values of
a drawing message are extracted from an enum for further processing,
with the bindings to the enum's fields established to the left of the
fat arrow (\lstinline[language=Rust]{=>}) and the block of code operating
on those bindings to the right.

\begin{figure}
\begin{lstlisting}
enum Canvas2dMsg {
  BeginPath,
  ArcTo(Point2D<f32>, Point2D<f32>, f32),
  EndPath
}

let msg = Canvas2dMsg::BeginPath;

match msg {
  BeginPath => { }
  EndPath => { }
  ArcTo(point_a, point_b, radius) => {
    draw_arc_to(point_a, point_b, radius);
  }
}
\end{lstlisting}
  \caption{Pattern matching to access the fields of an enum variant.}
  \label{fig:enums}
\end{figure}

Both enums and structs (which are substantially similar to C structs)
may have associated instance methods and static methods, called with the dot
(\lstinline[language=Rust]{.}) operator and the double-colon
(\lstinline[language=Rust]{::}) operator, respectively.
Static methods are common in Rust as they are used to construct
values, conventionally with a method called
\lstinline[language=Rust]{new}, as in
\lstinline[language=Rust]{Ponit2D::new(0.0, 0.0)}.
Notably, methods in Rust are dispatched statically, making them
eligable for inlining and aggressive optimization; methods are never
dispatched dynamically through a function pointer in Rust unless
explicitly requested.

Like many modern languages, Rust has first-class closures,
anonymous functions that capture their environment. These
use a compact notation with the arguments between pipes,
and the body between brackets: \lstinline[language=Rust]~|a, b| { a + b }~.
The closure that takes no arguments and performs no computation is
thus signified by the adorable series of characters, \lstinline[language=Rust]~||{}~.
Rust closures are translated as efficiently as other types
in Rust and can be inlined and optimized as well as any function.

Finally, Rust has a hygeinic macro system, much more powerful than
the C preprocessor. Macro invocations look like function invocations
where the function name is appended with a bang (\lstinline[language=Rust]{!}).
They are not functions though, but are expanded in place at compile time,
and perform syntactic transformations unavailable to functions.
The common method of performing console output, \lstinline[language=Rust]~println!(``{}'', foo)~,
is a macro.
See \figref{fig:bad-concurrency} for an example of closures and macros.

\subsection{Ownership}
The features of Rust described so far are common to many languages.
Rust's novelty though is in \emph{ownership and borrowing}.
In Rust, each value is uniquly owned by a single variable, and
assigning that value to another transfers ownership of, and access to,
that value to the receiver. Rust calls this transfer of ownership a \emph{move}.
When a value is allowed to go out of scope
it is destroyed. In this way ownership-based resource management is
like RAII\footnote{``resource acquisition is initialization''} in C++.

To make using variables usably ergonomic Rust can create temporary
references to values with the ampersand (\lstinline[language=Rust]{&})
operator. References are simply pointers with limited scope, and may be either
immutable or mutable. They are the only type of pointer in safe Rust.
\emph{The rules for borrowing ensure that no two pointers are ever mutably
aliased, and that invalid memory is never accessible.
This is what makes Rust memory safe.} See the example in \figref{fig:own}.

\begin{figure}
\begin{lstlisting}

// The single value we'll work with.
let val = Box::foo(0);

// Move the value in a to a new location
let moved_val = val;
// `val` is no longer accessible.
// The following will not compile.
// println(``{}'', val);

// Borrow and share the value immutably
// several times. This is done in a new
// block to limit the scope of the borrows.
{
  let shared1 = &moved_val;
  let shared2 = &moved_val;
  println!(``{}'', shared1);
  println!(``{}'', shared2);
  // Borrows are released at the end
  // of the block.
}
\end{lstlisting}
  \caption{Ownership and borrowing.}
  \label{fig:own}
\end{figure}

Ownership-based type systems are a kind of \emph{affine} type system. Rust's
model is influenced by the ownership model of Singularity OS~\cite{singularity},
as well as the region systems in the Cyclone language~\cite{cyclone} and MLKit~\cite{mlkit}.

\subsection{Ownership and concurrency}

Because the Rust type system provides very strong guarantees about memory aliasing,
Rust code is memory safe even in concurrent and multithreaded environments,
and, perhaps surprisingly, is even guaranteed to be \emph{data-race free}.

In concurrent programs, the data operated on by distinct threads is also itself distinct:
under Rust's ownership model, data cannot be owned by two threads at the same time.
For example, the code in \figref{fig:bad-concurrency} generates a static error from the compiler because
after the first thread is spawned, the ownership of \lstinline{data} has been transferred into the
closure associated with that thread and is no longer available in the original thread.
%http://is.gd/tkcD9R
\begin{figure}
\begin{lstlisting}
fn main() {
  // An owned pointer to a heap-allocated
  // integer
  let mut data = Box::new(0);

  // The `move` keyword here moves ownership of
  // the environment into the closure.
  thread::spawn(move || {
    *data = *data + 1;
  });
  // error: accessing moved value
  print!("{}", data);
}
\end{lstlisting}
  \caption{Code that will not compile because it attempts to access mutable state from two threads.}
  \label{fig:bad-concurrency}
\end{figure}

On the other hand, the immutable value in \figref{fig:shared-concurrency} can be borrowed and shared between multiple threads as long as those threads don't outlive the scope of the data, and even mutable values can be shared as long
as they are owned by a type that preserves the invariant that mutable memory is unaliased, as with the
mutex in \figref{fig:shared-mutable-concurrency}.

\begin{figure}
\begin{lstlisting}
fn main() {
  // An immutable shared pointer
  // (atomically reference counted)
  let data = Arc::new(1);
  let shared_data = data.clone();

  thread::spawn(move || {
    println!("{}", *shared_data);
  });
  print!("{}", *data);
}
\end{lstlisting}
  \caption{Safely reading immutable state from two threads.}
  \label{fig:shared-concurrency}
\end{figure}

\begin{figure}
\begin{lstlisting}
fn main() {
  // A heap allocated integer protected by an
  // atomically-reference-counted mutex
  let data = Arc::new(Mutex::new(0));
  let shared_data = data.clone();

  thread::spawn(move || {
    *shared_data.lock().unwrap() = 1;
  });

  print!("{}", *data.lock().unwrap());
}
\end{lstlisting}
  \caption{Safely mutating state from two threads.}
  \label{fig:shared-mutable-concurrency}
\end{figure}

With relatively few simple rules, ownership in Rust enables foolproof task parallelism,
but also data parallelism, by partitioning vectors and lending mutable references across threads.
Rust's concurrency abstractions are entirely implemented in libraries, and though
many advanced concurrent patterns such as work-stealing~~\cite{blumeofe:multiprogrammed-work-stealing}
cannot be implemented in safe Rust, they can usually be encapsulated in a memory-safe interface.

%%% Local Variables: 
%%% mode: latex
%%% TeX-master: "paper"
%%% End: 
