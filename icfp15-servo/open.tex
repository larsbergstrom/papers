%!TEX root = paper.tex

\section{Open}
\label{sec:open}
While this work has discussed many challenges in browser design and our current progress,
there are many other interesting open problems.

\paragraph{Just-in-time code} is produced from all modern JavaScript engines in order to
dynamically produce native code that is intended to execute more efficiently than an
interpreted strategy.
Unfortunately, this area is a large source of security bugs.
These bugs come from two sources.
First, there are potential correctness issues.
Many of these optimizations are only valid when certain conditions of the calling
code and environment hold, and ensuring the specialized code is called only when those
conditions hold is non-trivial to verify.
Second, dynamically producing and compiling native code and patching it into memory
while respecting all of the invariants required by the JavaScript runtime (e.g., the
garbage collector's tracing hooks or free vs. in-use registers) is also a challenge.

\paragraph{JavaScript object rooting} is a problem related to objects from the JavaScript
heap that are held onto by \Cplusplus object --- and sometimes vice-versa.
Today, all modern engines manually manage the boundary code to handle tracing
these objects, and this leads to leaking objects or, worse, freeing objects that
are still in use.
The lack of a typical garbage collector for these objects also means that cycles
between \Cplusplus and JavaScript need to be broken through ad-hoc means.

%%% Local Variables: 
%%% mode: latex
%%% TeX-master: "paper"
%%% End: 
