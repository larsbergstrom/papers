%!TEX root = paper.tex

\section{Rust}
\label{sec:rust}

TODO: basic overview, covering systems language basis; borrowing/ownership \& region/MLKit connection; ADT \& pattern matching; traits \& generics


\subsection{Concurrency and parallelism}

Rust contains library functionality for concurrency and data parallelism.
Concurrency, in particular, integrates cleanly with Rust's ownership types.
In safe concurrent programs, the data operated on by distinct threads is also itself distinct.
Through the use of Rust's ownership model, data cannot be owned by two threads at the same time.
For example, the following code generates a static error from the compiler because after the first
thread is spawned, the ownership of \lstinline{data} has been transferred into the closure associated
with that thread and is unavailable for transfer in to the closure for the second thread.
%http://is.gd/tkcD9R
\begin{lstlisting}
fn main() {
  let mut data = Box::new(0);

  Thread::spawn(move || {
    *data = *data + 1;
  });
  Thread::spawn(move || {
    *data = *data + 1;
  });
  print!(``{}'', data);
}
\end{lstlisting}

Data parallelism is a bit more tricky today because there is no way to \eg{} partition a vector in safe code
for a data parallel operation and then stitch it back together afterwards without performing a copy of the data.


%%% Local Variables: 
%%% mode: latex
%%% TeX-master: "paper"
%%% End: 
