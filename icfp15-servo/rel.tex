%!TEX root = paper.tex
%
\section{Related browser research}
\label{sec:rel}
The ZOOMM browser was an effort at Qualcomm Research to build a parallel browser, also
focused on multicore mobile devices~\cite{ZOOMM}.
This browser includes many features that we have not yet implemented in Servo, particularly
around resource fetching.
They also wrote their own JavaScript engine, which we have not done.
Servo and ZOOMM share an extremely concurrent architecture --- both have script, layout,
rendering, and the user interface operating concurrently from one another in order to maximize
interactivity for the user.
Parallel layout in Servo is one major area that was not investigated in the ZOOMM browser.
The other major difference is that Servo is implemented in Rust, whereas ZOOMM is written
in \Cplusplus, similarly to most modern browser engines.

Ras Bodik's group at the University of California Berkeley worked on a parallel browsing
project (funded in part by Mozilla Research) that focused on improving the parallelism
of layout~\cite{parallel-layout}.
Instead of our approach to parallel layout, which focuses on multiple parallel tree
traversals, they modeled a subset of CSS using attribute grammars.
They showed significant speedups with their system over a reimplementation of Safari's
algorithms, but we have not used this approach due to questions of whether it is possible
to use attribute grammars to both accurately model the web as it is implemented today
and to support new features as they are added.
Servo uses a very similar CSS selector matching algorithm to theirs.
